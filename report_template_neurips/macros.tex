% !TEX root = mom-robust.tex

\newtheorem{lem}{Lemma}

\newtheorem{thm}{Theorem}
\newenvironment{thmbis}[1]
{\renewcommand{\thethm}{\ref{#1}$'$}%
    \addtocounter{thm}{-1}%
    \begin{thm}}
        {\end{thm}}



\newcommand\tagthis{\addtocounter{equation}{1}\tag{\theequation}}

% Paired notation: usage explained below using \inp as an example:
% \inp just prints standard sized brackets and \inp* uses \left...\right to scale
% the brackets to enclose the material.
% Often \inp* will produce brackets that are too big, and manual scaling can be
% provided by \[\big], \[\Big], \[\bigg], \[\Bigg]
\DeclarePairedDelimiterX{\inp}[2]{\langle}{\rangle}{#1, #2}
%\DeclarePairedDelimiterX{\abs}[1]{\lvert}{\rvert}{#1}
\DeclarePairedDelimiterX{\norm}[1]{\lVert}{\rVert}{#1}
\DeclarePairedDelimiterX{\cbr}[1]{\{}{\}}{#1} % curly bracket
\DeclarePairedDelimiterX{\rbr}[1]{(}{)}{#1} % round bracket
\DeclarePairedDelimiterX{\sbr}[1]{[}{]}{#1} % square bracket

\providecommand{\refLE}[1]{\ensuremath{\stackrel{(\ref{#1})}{\leq}}}
\providecommand{\refEQ}[1]{\ensuremath{\stackrel{(\ref{#1})}{=}}}
\providecommand{\refGE}[1]{\ensuremath{\stackrel{(\ref{#1})}{\geq}}}
\providecommand{\refID}[1]{\ensuremath{\stackrel{(\ref{#1})}{\equiv}}}

\providecommand{\mgeq}{\succeq}
\providecommand{\mleq}{\preceq}

\providecommand{\divides}{\mid} % a divides b means there exists integer c such that b = ac
\providecommand{\tsum}{\textstyle\sum} % smaller sum symbols---displays as if inline
% basic sets
\providecommand{\R}{\mathbb{R}} % Reals
\providecommand{\real}{\mathbb{R}} % Reals
\providecommand{\N}{\mathbb{N}} % Naturals

% random variables
\DeclarePairedDelimiter{\paren}{(}{)}
\DeclareMathOperator{\expect}{\mathbb{E}}
\DeclareMathOperator{\E}{\expect}
\newcommand{\expec}[2][\!]{\mathbb{\expect}_{#1\!}\left[#2\right]}\DeclareMathOperator{\ind}{\mathbb{1}}
\DeclareMathOperator{\prob}{Pr}


% \sign on its own prints sign, additionally takes sign*[y]{x} and the output
% is sign(x), where y and star control the size of the rbr.
\DeclareMathOperator{\sgn}{sign}
\makeatletter
\def\sign{\@ifnextchar*{\@sgnargscaled}{\@ifnextchar[{\sgnargscaleas}{\@ifnextchar{\bgroup}{\@sgnarg}{\sgn} }}}
\def\@sgnarg#1{\sgn\rbr{#1}}
\def\@sgnargscaled#1{\sgn\rbr*{#1}}
\def\@sgnargscaleas[#1]#2{\sgn\rbr[#1]{#2}}
\makeatother

\DeclareMathOperator*{\argmin}{arg\,min}
\DeclareMathOperator*{\argmax}{arg\,max}
\DeclareMathOperator*{\supp}{supp}
\DeclareMathOperator*{\diag}{diag}
\DeclareMathOperator*{\Tr}{Tr}
\DeclareMathOperator*{\lin}{lin}


% bold vectors
\providecommand{\0}{\bm{0}}
\providecommand{\1}{\bm{1}}
\providecommand{\alphav}{\bm{\alpha}}
\renewcommand{\aa}{\bm{a}}
\providecommand{\bb}{\bm{b}}
\providecommand{\cc}{\bm{c}}
\providecommand{\dd}{\bm{d}}
\providecommand{\ee}{\bm{e}}
\providecommand{\ff}{\bm{f}}
\let\ggg\gg
\renewcommand{\gg}{\bm{g}}
\providecommand{\hh}{\bm{h}}
\providecommand{\ii}{\bm{i}}
\providecommand{\jj}{\bm{j}}
\providecommand{\kk}{\bm{k}}
\let\lll\ll
\renewcommand{\ll}{\bm{l}}
\providecommand{\mm}{\bm{m}}
\providecommand{\nn}{\bm{n}}
\providecommand{\oo}{\bm{o}}
\providecommand{\pp}{\bm{p}}
\providecommand{\qq}{\bm{q}}
\providecommand{\rr}{\bm{r}}
\let\sss\ss
\renewcommand{\ss}{\bm{s}}
\providecommand{\tt}{\bm{t}}
\providecommand{\uu}{\bm{u}}
\providecommand{\vv}{\bm{v}}
\providecommand{\ww}{\bm{w}}
\providecommand{\xx}{\bm{x}}
\providecommand{\yy}{\bm{y}}
\providecommand{\zz}{\bm{z}}
\providecommand{\thth}{\bm{\theta}}
\newcommand{\bxi}{\boldsymbol{\xi}}
\newcommand{\bmu}{\boldsymbol{\mu}}
\newcommand{\muv}{\bmu}

% tilde vectors
\providecommand{\txx}{\tilde\xx}
\providecommand{\tgg}{\tilde\gg}
\newcommand{\Var}{\mathrm{Var}}
\newcommand{\standarderror}{\mathrm{SE}}
\newcommand{\Cov}{\mathrm{Cov}}
\newcommand{\KL}{D_{\mathrm{KL}}}
\def\mSigma{{\bm{\Sigma}}}


% bold matrices
\providecommand{\mA}{\bm{A}}
\providecommand{\mB}{\bm{B}}
\providecommand{\mC}{\bm{C}}
\providecommand{\mD}{\bm{D}}
\providecommand{\mE}{\bm{E}}
\providecommand{\mF}{\bm{F}}
\providecommand{\mG}{\bm{G}}
\providecommand{\mH}{\bm{H}}
\providecommand{\mI}{\bm{I}}
\providecommand{\mJ}{\bm{J}}
\providecommand{\mK}{\bm{K}}
\providecommand{\mL}{\bm{L}}
\providecommand{\mM}{\bm{M}}
\providecommand{\mN}{\bm{N}}
\providecommand{\mO}{\bm{O}}
\providecommand{\mP}{\bm{P}}
\providecommand{\mQ}{\bm{Q}}
\providecommand{\mR}{\bm{R}}
\providecommand{\mS}{\bm{S}}
\providecommand{\mT}{\bm{T}}
\providecommand{\mU}{\bm{U}}
\providecommand{\mV}{\bm{V}}
\providecommand{\mW}{\bm{W}}
\providecommand{\mX}{\bm{X}}
\providecommand{\mY}{\bm{Y}}
\providecommand{\mZ}{\bm{Z}}
\providecommand{\mLambda}{\bm{\Lambda}}

% calligraphic
\providecommand{\cA}{\mathcal{A}}
\providecommand{\cB}{\mathcal{B}}
\providecommand{\cC}{\mathcal{C}}
\providecommand{\cD}{\mathcal{D}}
\providecommand{\cE}{\mathcal{E}}
\providecommand{\cF}{\mathcal{F}}
\providecommand{\cG}{\mathcal{G}}
\providecommand{\cH}{\mathcal{H}}
\providecommand{\cII}{\mathcal{H}}
\providecommand{\cJ}{\mathcal{J}}
\providecommand{\cK}{\mathcal{K}}
\providecommand{\cL}{\mathcal{L}}
\providecommand{\cM}{\mathcal{M}}
\providecommand{\cN}{\mathcal{N}}
\providecommand{\cO}{\mathcal{O}}
\providecommand{\cP}{\mathcal{P}}
\providecommand{\cQ}{\mathcal{Q}}
\providecommand{\cR}{\mathcal{R}}
\providecommand{\cS}{\mathcal{S}}
\providecommand{\cT}{\mathcal{T}}
\providecommand{\cU}{\mathcal{U}}
\providecommand{\cV}{\mathcal{V}}
\providecommand{\cX}{\mathcal{X}}
\providecommand{\cY}{\mathcal{Y}}
\providecommand{\cW}{\mathcal{W}}
\providecommand{\cZ}{\mathcal{Z}}

\providecommand{\error}{\mathcal{E}}
\providecommand{\ce}{\mathcal{C}}
\providecommand{\tce}{\Xi}

%%%%%%%%%%%%%%%%%%%%%%%%%
%%%%%% THEOREMS
%%%%%%%%%%%%%%%%%%%%%%%%%

% Theorems, propositions, observations, corollaries, conjectures
% , and hypotheses all have the same counter.
% Lemmas, claims, remarks, examples and properties have same counter.
% Definitions. notations and Assumptions have same alphabetic counter.

\newcommand{\QED}{\hfill $\square$}
\newtheorem{theorem}{Theorem}
\renewcommand*{\thetheorem}{\Roman{theorem}}
%{\theoremstyle{break}\newtheorem{thm}[theorem]{Theorem}}
\newtheorem{proposition}[theorem]{Proposition}
\newtheorem{observation}[theorem]{Observation}
\newtheorem{corollary}[theorem]{Corollary}
\newtheorem{hypothesis}[theorem]{Hypothesis}
\newtheorem{conjecture}[theorem]{Conjecture}
\newtheorem{claim}[theorem]{Claim}


\newtheorem{lemma}{Lemma}
\renewcommand*{\thelemma}{\arabic{lemma}}
\newtheorem{remark}[lemma]{Remark}
\newtheorem{example}[lemma]{Example}
\newtheorem{property}[lemma]{Property}

\newtheorem{definition}{Definition}
\renewcommand*{\thedefinition}{\Alph{definition}}
\newtheorem{convention}[definition]{Convention}
\newtheorem{notation}[definition]{Notation}

% ---------------------------------------------------------------------------- %
%                  Example: Customizing Assumption environment                 %
% ---------------------------------------------------------------------------- %
% Define a new style
\newtheoremstyle{shortassumptionstyle}%
{}% space above
{}% space below
{\itshape}% body font
{0.5em}% indent amount
{\bfseries}% theorem head font
{.}% punctuation after theorem head
{0.5em}% space after theorem head
{(\thmname{#1}\thmnumber{#2}) \thmnote{#3}}% theorem head spec
\theoremstyle{shortassumptionstyle}
\newtheorem{assumption}{A}
\renewcommand*{\theassumption}{\arabic{assumption}}
\crefname{assumption}{}{}
\Crefname{assumption}{}{}
\creflabelformat{assumption}{(#2A#1#3)}
\theoremstyle{plain}


\newcommand{\eg}{e.g.}
\newcommand{\ie}{i.e.}
\newcommand{\iid}{i.i.d.}
\newcommand{\cf}{cf.}

\newcommand{\h}{\widehat}
\newcommand{\ti}{\widetilde}
\newcommand{\ov}{\overline}
\newcommand{\wt}{\widetilde}
\newcommand{\set}[2][]{#1 \{ #2 #1 \} }
\newcommand{\ignore}[1]{}

\newcommand{\tgamma}{\tilde \gamma}
\newcommand{\true}{\texttt{true}}
\newcommand{\false}{\texttt{false}}
\newcommand{\e}{\epsilon}
% \usepackage[colorlinks=true,linkcolor=blue]{hyperref}
% \usepackage[capitalize,noabbrev]{cleveref}

\definecolor{color1}{RGB}{228,26,28}
\definecolor{color2}{RGB}{55,126,184}
\definecolor{color3}{RGB}{77,175,74}
\definecolor{color4}{RGB}{152,78,163}
\definecolor{color5}{RGB}{255,127,0}

% Check marks
\usepackage{pifont}
\newcommand{\cmark}{\ding{51}}%
\newcommand{\xmark}{\ding{55}}%
\newcommand{\yes}{$\checkmark$}%
\newcommand{\no}{$\times$}%

\newcommand{\minus}{\scalebox{0.8}{$-$}}
\newcommand{\plus}{\scalebox{0.6}{$+$}}

% custom item in enumerate with reference
\makeatletter
\newcommand{\myitem}[1]{%
    \item[\textbf{(#1)}]\protected@edef\@currentlabel{#1}%
}
\makeatother


% code to highlight parts of algorithm taken from https://tex.stackexchange.com/questions/386272/how-to-highlight-sections-of-my-code-in-algorithm
\usetikzlibrary{fit,calc}
%define a marking command
%define a marking command
\newcommand*{\tikzmk}[1]{\tikz[remember picture,overlay,] \node (#1) {};}
%define a boxing command, argument = color of box
\newcommand{\boxit}[1]{\tikz[remember picture,overlay]{\node[yshift=3pt,fill=#1,opacity=.25,fit={($(A)+(-0.2*\linewidth - 3pt,3pt)$)($(B)+(0.75*\linewidth - 5pt,-2pt)$)}] {};}}

\newcommand{\blockfed}[1]{\tikz[remember picture,overlay]{\node[yshift=3pt,fill=#1,opacity=.25,fit={($(A)+(-0.1*\linewidth - 3pt,3pt)$)($(B)+(0.88*\linewidth - 5pt,-2pt)$)}] {};}}

\newcommand{\highlight}[1]{\tikz[remember picture,overlay]{\node[yshift=3pt,fill=#1,opacity=.25,fit={($(A)+(2pt,6pt)$)($(B)+(-3pt,-7pt)$)}] {};}}
%define some colors according to algorithm parts (or any other method you like)
\colorlet{worker}{red!40}
\newcommand{\speedup}[1]{{\color{gray}(\ifdim #1 pt > 0.3pt #1\else $< #1$\fi{}$\times$)}}
% \newcommand{\speedup}[1]{{\color{lightgray} (#1 \times)}}
% \colorlet{worker}{cyan!60}


% Random variables
\def\reta{{\textnormal{$\eta$}}}
\def\ra{{\textnormal{a}}}
\def\rb{{\textnormal{b}}}
\def\rc{{\textnormal{c}}}
\def\rd{{\textnormal{d}}}
\def\re{{\textnormal{e}}}
\def\rf{{\textnormal{f}}}
\def\rg{{\textnormal{g}}}
\def\rh{{\textnormal{h}}}
\def\ri{{\textnormal{i}}}
\def\rj{{\textnormal{j}}}
\def\rk{{\textnormal{k}}}
\def\rl{{\textnormal{l}}}
% rm is already a command, just don't name any random variables m
\def\rn{{\textnormal{n}}}
\def\ro{{\textnormal{o}}}
\def\rp{{\textnormal{p}}}
\def\rq{{\textnormal{q}}}
\def\rr{{\textnormal{r}}}
\def\rs{{\textnormal{s}}}
\def\rt{{\textnormal{t}}}
\def\ru{{\textnormal{u}}}
\def\rv{{\textnormal{v}}}
\def\rw{{\textnormal{w}}}
\def\rx{{\textnormal{x}}}
\def\ry{{\textnormal{y}}}
\def\rz{{\textnormal{z}}}


% Random vectors
\def\rvepsilon{{\mathbf{\epsilon}}}
\def\rvtheta{{\mathbf{\theta}}}
\def\rva{{\mathbf{a}}}
\def\rvb{{\mathbf{b}}}
\def\rvc{{\mathbf{c}}}
\def\rvd{{\mathbf{d}}}
\def\rve{{\mathbf{e}}}
\def\rvf{{\mathbf{f}}}
\def\rvg{{\mathbf{g}}}
\def\rvh{{\mathbf{h}}}
\def\rvu{{\mathbf{i}}}
\def\rvj{{\mathbf{j}}}
\def\rvk{{\mathbf{k}}}
\def\rvl{{\mathbf{l}}}
\def\rvm{{\mathbf{m}}}
\def\rvn{{\mathbf{n}}}
\def\rvo{{\mathbf{o}}}
\def\rvp{{\mathbf{p}}}
\def\rvq{{\mathbf{q}}}
\def\rvr{{\mathbf{r}}}
\def\rvs{{\mathbf{s}}}
\def\rvt{{\mathbf{t}}}
\def\rvu{{\mathbf{u}}}
\def\rvv{{\mathbf{v}}}
\def\rvw{{\mathbf{w}}}
\def\rvx{{\mathbf{x}}}
\def\rvy{{\mathbf{y}}}
\def\rvz{{\mathbf{z}}}

% Random matrices
\def\rmA{{\mathbf{A}}}
\def\rmB{{\mathbf{B}}}
\def\rmC{{\mathbf{C}}}
\def\rmD{{\mathbf{D}}}
\def\rmE{{\mathbf{E}}}
\def\rmF{{\mathbf{F}}}
\def\rmG{{\mathbf{G}}}
\def\rmH{{\mathbf{H}}}
\def\rmI{{\mathbf{I}}}
\def\rmJ{{\mathbf{J}}}
\def\rmK{{\mathbf{K}}}
\def\rmL{{\mathbf{L}}}
\def\rmM{{\mathbf{M}}}
\def\rmN{{\mathbf{N}}}
\def\rmO{{\mathbf{O}}}
\def\rmP{{\mathbf{P}}}
\def\rmQ{{\mathbf{Q}}}
\def\rmR{{\mathbf{R}}}
\def\rmS{{\mathbf{S}}}
\def\rmT{{\mathbf{T}}}
\def\rmU{{\mathbf{U}}}
\def\rmV{{\mathbf{V}}}
\def\rmW{{\mathbf{W}}}
\def\rmX{{\mathbf{X}}}
\def\rmY{{\mathbf{Y}}}
\def\rmZ{{\mathbf{Z}}}

% Sets
\def\sA{{\mathbb{A}}}
\def\sB{{\mathbb{B}}}
\def\sC{{\mathbb{C}}}
\def\sD{{\mathbb{D}}}
% Don't use a set called E, because this would be the same as our symbol
% for expectation.
\def\sF{{\mathbb{F}}}
\def\sG{{\mathbb{G}}}
\def\sH{{\mathbb{H}}}
\def\sI{{\mathbb{I}}}
\def\sJ{{\mathbb{J}}}
\def\sK{{\mathbb{K}}}
\def\sL{{\mathbb{L}}}
\def\sM{{\mathbb{M}}}
\def\sN{{\mathbb{N}}}
\def\sO{{\mathbb{O}}}
\def\sP{{\mathbb{P}}}
\def\sQ{{\mathbb{Q}}}
\def\sR{{\mathbb{R}}}
\def\sS{{\mathbb{S}}}
\def\sT{{\mathbb{T}}}
\def\sU{{\mathbb{U}}}
\def\sV{{\mathbb{V}}}
\def\sW{{\mathbb{W}}}
\def\sX{{\mathbb{X}}}
\def\sY{{\mathbb{Y}}}
\def\sZ{{\mathbb{Z}}}


\renewcommand{\epsilon}{\varepsilon}

\newcommand{\tofix}[1]{{\color{red} (ToFix: #1)}}
\newcommand{\todo}[1]{{\color{blue} (Todo: #1)}}
\newcommand{\comment}[1]{{\color{brown} (Djian: #1)}}

% Set-theoretic difference
\newcommand\rsetminus{\mathbin{\mathpalette\rsetminusaux\relax}}
\newcommand\rsetminusaux[2]{\mspace{-4mu}
  \raisebox{\rsmraise{#1}\depth}{\rotatebox[origin=c]{-20}{$#1\smallsetminus$}}
 \mspace{-4mu}
}
\newcommand\rsmraise[1]{%
  \ifx#1\displaystyle .8\else
    \ifx#1\textstyle .8\else
      \ifx#1\scriptstyle .6\else
        .45%
      \fi
    \fi
  \fi}